\chapter*{ВВЕДЕНИЕ}
\addcontentsline{toc}{chapter}{ВВЕДЕНИЕ}

Расстояние Левенштейна --- минимальное количество редакторских операций вставки, удаления и замены символа, которое необходимо для преобрзования одной строки в другую. 

Впервые задачу поставил в 1965 году советский математик Владимир Левенштейн \cite{Levenshtein} при изучении последовательностей нулей и единиц, впоследствии более общую задачу для произвольного алфавита связали с его именем. Большой вклад в изучение вопроса внёс Дэн Гасфилд.

Расстояние Левенштейна применяется в следующих областях. 
\begin{enumerate}[label={\arabic*)}]
	\item Компьютерная лингвистика для исправления ошибок.
	\item Сравнение текстовых файлов утилитой \code{diff} и  другими.
	\item Биоинформатика --- для сравнения генов, хромосом и белков.
\end{enumerate}

Расстояние Дамерау-Левенштейна (названо в честь учёных Фредерика Дамерау и Владимира Левенштейна) --- минимальное количество редакторских операций вставки, удаления, замены символа или транспозиции символов, которое необходимо для преобразования одной строки в другую.

Цель лабораторной работы --- изучение, анализ и реализация алгоритмов нахождения расстояний между строками Левенштейна и Дамерау-Левенштейна.

Задачи лабораторной работы:

\begin{itemize}
	\item[---] изучить алгоритмы Левенштейна и Дамерау-Левенштейна;
	\item[---] применить методы динамического программирования для реализации алгоритмов;
	\item[---] получить практические навыки реализации алгоритмов Левенштейна и Дамерау-Левенштейна;
	\item[---] провести сравнительный анализ на основе экспериментальных данных;
	\item[---] подготовить отчет по лабораторной работе.
\end{itemize}

\pagebreak
