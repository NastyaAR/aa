\chapter*{ВВЕДЕНИЕ}
\addcontentsline{toc}{chapter}{ВВЕДЕНИЕ}

\textbf{Матрицами} называются массивы элементов, представленные в виде прямоугольных таблиц, для которых определены правила математических действий. Элементами матрицы могут являться числа, алгебраические символы или математические функции \cite{Matrix}.

Матричная алгебра имеет обширные применения в различных отраслях знания – в математике, физике, информатике, экономике. Например, матрицы используется для решения систем алгебраических и дифференциальных уравнений, нахождения значений физических величин в квантовой теории, шифрования сообщений в Интернете \cite{Matrix}.

Цель лабораторной работы ---  изучение, реализация и сравнение алгоритмов
умножения матриц. В данной лабораторной работе рассматриваются стандартный алгоритм умножения матриц (с оптимизациями и без), алгоритм Винограда умножения матриц (с оптимизациями и без), алгоритм Штрассена умножения матриц (с оптимизациями и без).

Задачи лабораторной работы:

\begin{itemize}
	\item[---] исследовать алгоритмы умножения матриц;
	\item[---] провести оптимизацию алгоритмов умножения матриц;
	\item[---] применить методы динамического программирования для реализации алгоритмов умножения матриц;
	\item[---] оценить и сравнить трудоёмкость алгоритмов;
	\item[---] провести сравнительный анализ по времени и памяти на основе экспериментальных данных;
	\item[---] подготовить отчет по лабораторной работе.
\end{itemize}

\pagebreak
