\chapter{Аналитическая часть}

Умножение матриц является основным инструментом линейной алгебры и имеет многочисленные применения в математике, физике, программировании \cite{article}. Одним из самых
эффективных по времени алгоритмов умножения матриц является алгоритм Винограда.

Пусть матрицы $A$ и $B$ заданы следующим образом:
\begin{equation}
	A = \left(
	\begin{array}{cccc}
		a_{11} & a_{12} & \ldots & a_{1n}\\
		a_{21} & a_{22} & \ldots & a_{2n}\\
		\vdots & \vdots & \ddots & \vdots\\
		a_{m1} & a_{m2} & \ldots & a_{mn}
	\end{array}
	\right),
\end{equation}

\begin{equation}
	B = \left(
	\begin{array}{cccc}
		b_{11} & b_{12} & \ldots & b_{1k}\\
		b_{21} & b_{22} & \ldots & b_{2k}\\
		\vdots & \vdots & \ddots & \vdots\\
		b_{n1} & b_{n2} & \ldots & b_{nk}
	\end{array}
	\right),
\end{equation}

где: 
\begin{itemize}
	\item[---] $m$ --- количество строк в матрице $A$;
	\item[---] $n$ --- количество столбцов в матрице $A$ и количество строк в матрице $B$;
	\item[---] $k$ --- количество столбцов в матрице $B$.
\end{itemize}

Количество столбцов матрицы $A$ должно быть равно количеству строк матрицы $B$, чтобы можно было осуществить умножение. При этом результирующая матрица $A \times B$ будет иметь размер $[m \times k]$. 

Каждый элемент результирующей матрицы (обозначим её $C$) вычисляется как:
\begin{equation}
	c_{ij} = \displaystyle\sum_{r=1}^{n} a_{ir} b_{rj},
\end{equation}

где:
\begin{itemize}
	\item[---] $i = 1 \ldots m$;
	\item[---] $j = 1 \ldots k$;
	\item[---] $r = 1 \ldots n$.
\end{itemize}

Алгоритмическая сложность классического алгоритма умножения матриц $O(n^3)$, так как необходимо использовать три цикла.

\section{Алгоритм Винограда умножения матриц}

Каждый элемент в результирующей матрице произведении матриц представляет собой скалярное произведение соответствующих строки и столбца исходных матриц. Можно заметить также, что такое умножение допускает предварительную обработку, позволяющую часть работы выполнить заранее, перед основными вложенными циклами умножения матриц.

Рассмотрим два вектора $V = (v_1, v_2, v_3, v_4)$ и $W = (w_1, w_2, w_3, w_4)$. Их скалярное произведение равно:
\begin{equation}
	V \cdot W = v_1w_1 + v_2w_2 + v_3w_3 + v_4w_4.
\end{equation}

Это равенство можно переписать в виде:
\begin{equation}
	V \cdot W = (v_1 + w_2)(v_2 + w_1) + (v_3 + w_4)(v_4 + w_3) - v_1v_2 - v_3v_4 - w_1w_2 - w_3w_4.
\end{equation}

Выражение в правой части последнего равенства допускает предварительную обработку: его части можно вычислить заранее и запомнить для каждой строки первой матрицы и для каждого столбца второй. 

Таким образом, количество умножений по сравнению с классически расписанных скалярным произведением векторов (4) уменьшено на два и составляет два умножения.

\section{Алгоритм Штрассена умножения матриц}

Алгоритм Штрассена – это разработанный Фолькером Штрассеном в 1969 году алгоритм, предназначенный для более быстрого умножения матриц. Он является обобщением метода умножения Карацубы на матрицы и предлагает более эффективный подход к умножению крупных матриц.

В отличие от классического алгоритма умножения, алгоритм Штрассена использует рекурсию и разделяет умножение матриц на подзадачи меньшего размера. Затем он комбинирует результаты рекурсивных вычислений, используя несколько предварительно вычисленных слагаемых. Это позволяет снизить число операций умножения и, в результате, улучшить время выполнения алгоритма.

Пусть $A, B$ --- матрицы размера $2^k \times 2^k$. Их можно представить как блочные матрицы $2 \times 2$ из $2^{k-1} \times 2^{k-1}$ матриц:

\begin{equation}
	A = \left(
	\begin{array}{cc}
		A_{11} & A_{12} \\
		A_{21} & A_{22} \\
	\end{array}
	\right),
\end{equation}

\begin{equation}
	B = \left(
	\begin{array}{cc}
		B_{11} & B_{12} \\
		B_{21} & B_{22} \\
	\end{array}
	\right),
\end{equation}

По принципу блочного умножения матрица $AB$ выражается следующим образом:

\begin{equation}
	AB = \left(
	\begin{array}{cc}
		A_{11}B_{11} + A_{12}B_{21} & A_{11}B_{12} + A_{12}B_{22}\\
		A_{21}B_{11} + A_{22}B_{21} & A_{21}B_{12} + A_{22}B_{22} \\
	\end{array}
	\right).
\end{equation}

Штрассен предложил модифицировать это выражение, чтобы получить семь умножений вместо восьми:

\begin{equation}D = (A_{11} + A_{22})(B_{11} + B_{22}),\end{equation}
\begin{equation}D_1 = (A_{12} - A_{22})(B_{21} + B_{22}),\end{equation}
\begin{equation}D_2 = (A_{21} - A_{11})(B_{11} + B_{12}),\end{equation}
\begin{equation}H_1 = (A_{11} + A_{12})B_{22},\end{equation}
\begin{equation}H_2 = (A_{21} + A_{22})B_{11},\end{equation}
\begin{equation}V_1 = A_{22}(B_{21} - B_{11}),\end{equation}
\begin{equation}V_2 = A_{11}(B_{12} - B_{22}),\end{equation}

\begin{equation}
	AB = \left(
	\begin{array}{cc}
		D+D_1+V_1-H_1 & V_2+H_1\\
		V_1 + H_2 & D+D_2+V_2-H_2 \\
	\end{array}
	\right).
\end{equation}

\section*{Вывод}

Были рассмотрены алгоритмы умножения матриц: классическое умножение матриц, алгоритм Винограда, алгоритм Штрассена. Основное двух последних алгоритмов в отличие от классического алгоритма умножения матриц состоит в выполнении некоторой части вычислений предварительно и, как следствие, в уменьшении количества умножений, которые является более дорогостоящей операцией в смысле трудоёмкости алгоритмов.

\pagebreak

