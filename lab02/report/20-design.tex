\chapter{Конструкторская часть}

\section{Разработка классического алгоритма умножения матриц}

На рисунке \ref{img:classic1} приведена схема классического алгоритма умножения матриц без оптимизаций.

\pdfimg{170mm}{classic1}{Схема классического алгоритма умножения матриц без оптимизаций}

\pagebreak

\section{Разработка классического алгоритма умножения матриц с оптимизацией}

На рисунке \ref{img:classic2} приведена схема классического алгоритма умножения матриц c заменой $a = a + b$ на $a += b$.

\pdfimg{170mm}{classic2}{Схема классического алгоритма умножения матриц c заменой $a = a + b$ на $a += b$}

\pagebreak

\section{Разработка алгоритма Винограда умножения матриц без оптимизаций}

На рисунках \ref{img:noopt1} --- \ref{img:noopt5} приведена схема алгоритма Винограда умножения матриц без оптимизаций.

\pdfimg{145mm}{noopt1}{Схема алгоритма Винограда умножения матриц без оптимизаций: начальная инициализация}

\pagebreak

\pdfimg{195mm}{noopt2}{Схема алгоритма Винограда умножения матриц без оптимизаций: предвычисление для строк матрицы}

\pagebreak

\pdfimg{195mm}{noopt3}{Схема алгоритма Винограда умножения матриц без оптимизаций: предвычисление для столбцов матрицы}

\pagebreak

\pdfimg{230mm}{noopt4}{Схема алгоритма Винограда умножения матриц без оптимизаций: основной цикл умножения}

\pagebreak

\pdfimg{205mm}{noopt5}{Схема алгоритма Винограда умножения матриц без оптимизаций: обработка нечётной размерности}

\pagebreak

\section{Разработка алгоритма Винограда умножения матриц с оптимизациями}

На рисунках \ref{img:grapeopt1} --- \ref{img:grapeopt5} приведена схема алгоритма Винограда умножения матриц с заменой умножения и деления на два на побитовый сдвиг, $a = a + b$ на $a += b$.

\pdfimg{145mm}{grapeopt1}{Схема оптимизированного алгоритма Винограда умножения матриц: начальная инициализация}

\pagebreak

\pdfimg{195mm}{grapeopt2}{Схема оптимизированного алгоритма Винограда умножения матриц: предвычисление для строк матрицы}

\pagebreak

\pdfimg{195mm}{grapeopt3}{Схема оптимизированного алгоритма Винограда умножения матриц: предвычисление для столбцов матрицы}

\pagebreak

\pdfimg{230mm}{grapeopt4}{Схема оптимизированного алгоритма Винограда умножения матриц: основной цикл умножения}

\pagebreak

\pdfimg{205mm}{grapeopt5}{Схема оптимизированного алгоритма Винограда умножения матриц: обработка нечётной размерности}

\pagebreak

\section{Разработка алгоритма Штрассена умножения матриц без оптимизаций}

На рисунках \ref{img:strnoopt} --- \ref{img:strnoopt2} приведена схема алгоритма Штрассена умножения матриц без оптимизаций.

\pdfimg{65mm}{strnoopt}{Схема классического алгоритма Штрассена умножения матриц}

\pdfimg{225mm}{strnoopt2}{Схема классического алгоритма Штрассена умножения матриц}

\pagebreak

\section{Разработка алгоритма Штрассена умножения матриц с оптимизациями}

На рисунках \ref{img:stropt} --- \ref{img:stropt2} приведена схема алгоритма Штрассена умножения матриц с оптимизациями: замена деления на два на побитовый сдвиг, предвычисление некоторых слагаемых.

\pdfimg{65mm}{stropt}{Схема оптимизированного алгоритма Штрассена умножения матриц}

\pdfimg{225mm}{stropt2}{Схема оптимизированного алгоритма Штрассена умножения матриц}

\pagebreak

\section{Трудоемкость алгоритмов}

\subsection{Модель вычислений}

Чтобы провести вычисление трудоемкости алгоритмов умножения матриц, введем модель вычислений:

\begin{enumerate}
	\item операции из списка (\ref{for:opers}) имеют трудоемкость 1;
	\begin{equation}
		\label{for:opers}
		+, -, ==, !=, <, >, <=, >=, [], ++, {-}-, +=, -=, =, \&\&, ||
	\end{equation}
	\item операции из списка (\ref{for:opers2}) имеют трудоемкость 2;
	\begin{equation}
		\label{for:opers2}
		*, /, *=, /=, \%
	\end{equation}
	\item трудоемкость оператора выбора \code{if условие then A else B} рассчитывается, как (\ref{for:if});
	\begin{equation}
		\label{for:if}
		f_{if} = f_{\text{условия}} +
		\begin{cases}
			f_A, & \text{если условие выполняется,}\\
			f_B, & \text{иначе.}
		\end{cases}
	\end{equation}
	\item трудоемкость цикла рассчитывается, как (\ref{for:for});
	\begin{equation}
		\label{for:for}
		f_{for} = f_{\text{инициализации}} + f_{\text{сравнения}} + N(f_{\text{тела}} + f_{\text{инкремента}} + f_{\text{сравнения}})
	\end{equation}
	\item трудоемкость вызова функции равна 0.
\end{enumerate}

\subsection{Трудоёмкость стандартного алгоритма умножения матриц}

Для стандартного алгоритма умножения матриц трудоемкость будет слагаться из:

\begin{itemize}
	\item[---] создания матрицы-результата, трудоёмкость которого: $f = 1 + 1 + (1 + 1 + M \cdot (1 + 1 + 2))$;
	\item[---] внешнего цикла по $i \in [0..M-1]$, трудоёмкость которого: $f = 1 + 1 + M \cdot (1 + 1 + f_{body})$;
	\item[---] цикла по $j \in [0..N-1]$, трудоёмкость которого: $f = 1 + 1 + N \cdot (1 + 1 + f_{body})$;
	\item[---] цикла по $k \in [0..K-1]$, трудоёмкость которого: $f = 1 + 1 + K \cdot (2 + 1 + 2 + 1 + 2 + 2 + 2 + 1 + 1) = 2 + 14K$.
\end{itemize}

При раскрытии скобок получается:

$f = 1 + 1 + (1 + 1 + M \cdot (1 + 1 + 2)) + 1 + 1 + M \cdot (1 + 1 + (1 + 1 + N \cdot (1 + 1 + 2 + 14K))) = 14MNK + 4MN + 8M + 6 \approx 14MNK$

Оценка трудоёмкости по самому быстрорастущему слагаемому из зависимости от входных данных: $O(n^3)$.

\subsection{Трудоёмкость оптимизированного алгоритма умножения матриц}

Для оптимизированного алгоритма умножения матриц трудоемкость будет слагаться из:

\begin{itemize}
	\item[---] создания матрицы-результата, трудоёмкость которого: $f = 1 + 1 + (1 + 1 + M \cdot (1 + 1 + 2))$;
	\item[---] внешнего цикла по $i \in [0..M-1]$, трудоёмкость которого: $f = 1 + 1 + M \cdot (1 + 1 + f_{body})$;
	\item[---] цикла по $j \in [0..N-1]$, трудоёмкость которого: $f = 1 + 1 + N \cdot (1 + 1 + f_{body})$;
	\item[---] цикла по $k \in [0..K-1]$, трудоёмкость которого: $f = 1 + 1 + K \cdot (2 + 1 + 2 + 2 + 2 + 1 + 1) = 2 + 11K$.
\end{itemize}

При раскрытии скобок получается:

$f = 1 + 1 + (1 + 1 + M \cdot (1 + 1 + 2)) + 1 + 1 + M \cdot (1 + 1 + (1 + 1 + N \cdot (1 + 1 + 2 + 11K))) = 11MNK + 4MN + 8M + 6 \approx 11MNK$

Оценка трудоёмкости по самому быстрорастущему слагаемому из зависимости от входных данных: $O(n^3)$.

\subsection{Трудоёмкость стандартного алгоритма Винограда умножения матриц}

Для стандартного алгоритма Винограда умножения матриц трудоемкость будет слагаться из:

\begin{itemize}
	\item[---] создания матрицы-результата, трудоёмкость которого: $f = 1 + 1 + (1 + 1 + M \cdot (1 + 1 + 2)) = 4M+4$;
	\item[---] начальной инициализации переменных и списков для вычисляемых значений для столбцов и строк, трудоёмкость которых: $f = 6 + 2 = 8$;
	\item[---] заполнения массивов rowFactor и columnFactor, трудоёмкость которых одинакова и составляет: $f = 2 \cdot (1 + 1 + N \cdot (1 + 1 + \frac{K}{2} \cdot (1 + 1 + 4 + 4 + 2 + 5))) = 17NK + 4N + 4$; 
	\item[---] внешнего цикла по $i \in [0..M-1]$, трудоёмкость которого: $f = 1 + 1 + M \cdot (1 + 1 + f_{body})$;
	\item[---] цикла по $j \in [0..N-1]$, трудоёмкость которого: $f = 1 + 1 + N \cdot (1 + 1 + 7 + f_{body})$;
	\item[---] цикла по $k \in [0..K/2-1]$, трудоёмкость которого: $f = 1 + 1 + \frac{K}{2} \cdot 30 = 2 + 15K$;
	\item[---] условия, трудоёмкость которого: $f = 2 + 1 + 0 = 3$, если условие не выполнилось; $f = 2 + 1 + 1 + 1 + M \cdot (1 + 1 + (1 + 1 + N \cdot (1 + 1 + 14))) = 16MN+4M+5$.
\end{itemize}

Итого, в случае невыполнения условия:

$f = 4M+4+8+17NK+4N+4+15MNK+11MN+4M+2+3=15MNK + 11MN+17NK+8M+21 \approx 15MNK$

Итого, в случае выполнения условия:

$f = 4M+4+8+17NK+4N+4+15MNK+11MN+4M+2+16MN+4M+5=15MNK + 11MN+12MN+17NK+12M+23 \approx 15MNK$

Оценка трудоёмкости по самому быстрорастущему слагаемому из зависимости от входных данных: $O(n^3)$.

\subsection{Трудоёмкость оптимизированного алгоритма Винограда умножения матриц}

Для оптимизированного алгоритма Винограда умножения матриц трудоемкость будет слагаться из:

\begin{itemize}
	\item[---] создания матрицы-результата, трудоёмкость которого: $f = 1 + 1 + (1 + 1 + M \cdot (1 + 1 + 2)) = 4M+4$;
	\item[---] начальной инициализации переменных и списков для вычисляемых значений для столбцов и строк, трудоёмкость которых: $f = 2 + 2 = 4$;
	\item[---] заполнения массивов rowFactor и columnFactor, трудоёмкость которых одинакова и составляет: $f = 2 \cdot (1 + 1 + N \cdot (1 + 1 + \frac{K}{2} \cdot (1 + 1 + 3 + 2 + 4))) = 11NK + 4N + 4$; 
	\item[---] внешнего цикла по $i \in [0..M-1]$, трудоёмкость которого: $f = 1 + 1 + M \cdot (1 + 1 + f_{body})$;
	\item[---] цикла по $j \in [0..N-1]$, трудоёмкость которого: $f = 1 + 1 + N \cdot (1 + 1 + 7 + f_{body})$;
	\item[---] цикла по $k \in [0..K/2-1]$, трудоёмкость которого: $f = 1 + 1 + \frac{K}{2} \cdot 22 = 2 + 11K$;
	\item[---] условия, трудоёмкость которого: $f = 2 + 1 + 0 = 3$, если условие не выполнилось; $f = 2 + 1 + 1 + 1 + M \cdot (1 + 1 + (1 + 1 + N \cdot (1 + 1 + 11))) = 13MN+4M+5$.
\end{itemize}

Итого, в случае невыполнения условия:

$f = 4M+4+4+11NK+4N+4+11MNK+11MN+4M+2+3 = 11MNK+11MN+11NK+8M+4N+17 \approx 11MNK$

Итого, в случае выполнения условия:

$f = 4M+4+4+11NK+4N+4+11MNK+11MN+4M+2 + 13MN+4M+5 = 11MNK+24MN+11NK+12M+4n+19 \approx 11MNK$

Оценка трудоёмкости по самому быстрорастущему слагаемому из зависимости от входных данных: $O(n^3)$.

\subsection{Трудоёмкость стандартного алгоритма Штрассена умножения матриц}

Трудоёмкость сложения/вычитания матриц складывается из:
\begin{itemize}
	\item[---] создания матрицы-результата, трудоёмкость которого: $f = 1 + 1 + (1 + 1 + M \cdot (1 + 1 + 2)) = 4M+4$;
	\item[---] двух циклов, вложенных один в другой, трудоёмкость которых: $f = 1 + 1 + M \cdot (1 + 1 + (1 + 1 + N \cdot (1 + 1 + 8))) = 10MN + 4M + 2$.
\end{itemize}

В итоге:

$f = 10MN + 8M + 6$

Для стандартного алгоритма Штрассена умножения матриц трудоемкость будет слагаться из:

\begin{itemize}
	\item[---] начальной инициализации переменной n, трудоёмкость которой: $f = 1$;
	\item[---] условия, трудоёмкость которого: $f = 1$, если условие не выполнилось; $f = 1 + 1 + 1 + (1 + 1 + 1 \cdot (1 + 1 + 2)) + 9 = 18$, если условие выполнилось;
	\item[---] создания 8 матриц-блоков, трудоёмкость которого: $f = 32 + 8 * (1 + 1 + (1 + 1 + \frac{M}{2} \cdot (1 + 1 + 2))) = 16M+64$;
	\item[---] инициализации 8 матриц-блоков, трудоёмкость которой: $f = 1 + 2 + \frac{M}{2} \cdot (2 + 1 + 60) = \frac{63}{2}M + 3$;  
	\item[---] инициализации переменных $p1 \ldots p10$, трудоёмкость которой: $f = 10 \cdot (10M^2 + 8M + 6 + 1) = 100M^2 + 80M + 70$;
	\item[---] присвоения переменным $k1 \ldots k7$ результата вызова рекурсивной функции, трудоёмкость которого: $f = 7$;
	\item[---] получения частей результирующей матрицы, трудоёмкость которого: $f = 8 + 8 * (10M^2 + 8M + 6) = 80M^2 + 64M + 56$;
	\item[---] создания матрицы-результата, трудоёмкость которого: $f = 1 + 1 + (1 + 1 + M \cdot (1 + 1 + 2)) = 4M+4$;
	\item[---] двух циклов заполнения результирующей матрицы, трудоёмкость которых: $f = 1 + 3 + \frac{M}{2} \cdot (3 + 1 + 4) + 3 + 1 + \frac{M}{2} \cdot (1 + 1 + 10) = 10M + 8$.
	
\end{itemize}

Алгоритм Штрассена использует рекурсию. Чтобы подсчитать общую трудоёмкость, необходимо вычислить количество рекурсивных вызовов. Матрица размером 2 на 2 выполнит одно разбиение на четыре матрицы размером один на один. Матрицы размером 1 на 1 не вызывают дальнейших рекурсивных вызовов, так как являются основанием рекурсии. Соответственно, произвольная матрица размером $2^n$ на $2^n$ будет иметь $7^n$ рекурсивных вызовов до достижения основания, трудоёмкость каждого из которых равна общей трудоёмкости при невыполнении условия:

$f = 7^M \cdot (1 + 16M+64 + \frac{63}{2}M + 3 + 100M^2 + 80M + 70 + 7 + 80M^2 + 64M + 56 + 4M+4 + 10M+8) = 213 \cdot 7^M + \frac{411 \cdot 7^M \cdot M}{2} + 180 \cdot 7^M \cdot M^2$.

По достижении основания рекурсии рекурсивных вызовов не будет выполнено:

$f = M^2 \cdot 18 = 18M^2$ 

Итого, трудоёмкость алгоритма Штрассена умножения матриц составляет:

$f = 213 \cdot 7^M + \frac{559 \cdot 7^M \cdot M}{2} + 100 \cdot 7^M \cdot M^2 + 18M^2$

Оценка трудоёмкости по самому быстрорастущему слагаемому из зависимости от входных данных: $O(n^2 \cdot 7^n)$.

\subsection{Трудоёмкость оптимизированного алгоритма Штрассена умножения матриц}

Трудоёмкость сложения/вычитания матриц складывается из:
\begin{itemize}
	\item[---] создания матрицы-результата, трудоёмкость которого: $f = 1 + 1 + (1 + 1 + M \cdot (1 + 1 + 2)) = 4M+4$;
	\item[---] двух циклов, вложенных один в другой, трудоёмкость которых: $f = 1 + 1 + M \cdot (1 + 1 + (1 + 1 + N \cdot (1 + 1 + 8))) = 10MN + 4M + 2$.
\end{itemize}

В итоге:

$f = 10MN + 8M + 6$

Для оптимизированного алгоритма Штрассена умножения матриц трудоемкость будет слагаться из:

\begin{itemize}
	\item[---] начальной инициализации переменной n, трудоёмкость которой: $f = 1$;
	\item[---] условия, трудоёмкость которого: $f = 1$, если условие не выполнилось; $f = 1 + 1 + 1 + (1 + 1 + 1 \cdot (1 + 1 + 2)) + 9 = 18$, если условие выполнилось;
	\item[---] создания 8 матриц-блоков, трудоёмкость которого: $f = 32 + 8 * (1 + 1 + (1 + 1 + \frac{M}{2} \cdot (1 + 1 + 2))) = 16M+56$;
	\item[---] инициализации 8 матриц-блоков, трудоёмкость которой: $f = 1 + 2 + \frac{M}{2} \cdot (2 + 1 + 48) = \frac{51}{2}M + 3$;  
	\item[---] инициализации переменных $p1 \ldots p10]$, трудоёмкость которой: $f = 10 \cdot (10M^2 + 8M + 6 + 1) = 100M^2 + 80M + 70$;
	\item[---] присвоения переменным $k1 \ldots k7$ результата вызова рекурсивной функции, трудоёмкость которого: $f = 7$;
	\item[---] получения частей результирующей матрицы, трудоёмкость которого: $f = 8 + 8 * (10M^2 + 8M + 6) = 80M^2 + 64M + 56$;
	\item[---] создания матрицы-результата, трудоёмкость которого: $f = 1 + 1 + (1 + 1 + M \cdot (1 + 1 + 2)) = 4M+4$;
	\item[---] двух циклов заполнения результирующей матрицы, трудоёмкость которых: $f = 1 + 2 + \frac{M}{2} \cdot (3 + 1 + 4) + 2 + 1 + \frac{M}{2} \cdot (1 + 1 + 8) = 9M + 6$.
	
\end{itemize}

Алгоритм Штрассена использует рекурсию. Чтобы подсчитать общую трудоёмкость, необходимо вычислить количество рекурсивных вызовов. Матрица размером 2 на 2 выполнит одно разбиение на четыре матрицы размером один на один. Матрицы размером 1 на 1 не вызывают дальнейших рекурсивных вызовов, так как являются основанием рекурсии. Соответственно, произвольная матрица размером $2^n$ на $2^n$ будет иметь $7^n$ рекурсивных вызовов до достижения основания, трудоёмкость каждого из которых равна общей трудоёмкости при невыполнении условия:

$f = 7^M \cdot (1 + 16M+64 + \frac{51}{2}M + 3 + 100M^2 + 80M + 70 + 7 + 80M^2 + 64M + 56 + 4M+4 + 9M+6) = 211 \cdot 7^M + \frac{397 \cdot 7^M \cdot M}{2} + 180 \cdot 7^M \cdot M^2$.

По достижении основания рекурсии рекурсивных вызовов не будет выполнено:

$f = M^2 \cdot 18 = 18M^2$ 

Итого, трудоёмкость алгоритма Штрассена умножения матриц составляет:

$f = 203 \cdot 7^M + \frac{397 \cdot 7^M \cdot M}{2} + 100 \cdot 7^M \cdot M^2 + 18M^2$

Оценка трудоёмкости по самому быстрорастущему слагаемому из зависимости от входных данных: $O(n^2 \cdot 7^n)$.

\section*{Вывод}

В данном разделе представлены схемы алгоритмов умножения матриц: классического алгоритма умножения матриц без оптимизаций, оптимизированного алгоритма умножения матриц, классического алгоритма Винограда, оптимизированного алгоритма Винограда, классического алгоритма Штрассена, оптимизированного алгоритма Штрассена, также оценена их трудоёмкость.

По итогам расчётов самыми менее трудоёмкими реализациями алгоритмов являются оптимизированный классический алгоритм умножения матриц и оптимизированный алгоритм Винограда умножения матриц. Алгоритм Штрассена в данной реализации является значительно более трудоёмким, чем остальные алгоритмы. Применение оптимизаций в каждом случае показало уменьшение трудоёмкости реализации алгоритма.

