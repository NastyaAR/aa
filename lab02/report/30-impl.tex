\chapter{Технологическая часть}

В данном разделе представлены требования к программному обеспечению, а также рассматриваются средства реализации и приводятся листинги кода.

\section{Требования к программному обеспечению}

К программе предъявляется ряд требований: на вход подаются указатели на две матрицы, на выходе --- матрица, являющаяся произведением исходных.

\section{Средства реализации}

В качестве языка программирования для реализации лабораторной работы был выбран многопоточный язык Gо \cite{golang}. Выбор был сделан в пользу данного языка программирования, вследствие наличия пакетов для тестирования программного обеспечения.

\section{Реализация}

В листингах \ref{lst:clnoopt}--\ref{lst:clopt} приведены реализации алгоритмов классического алгоритма умножения матриц и оптимизированного классического алгоритма умножения матриц.
\newpage

\begin{lstinputlisting}[
	caption={Классический алгоритм умножения матриц без оптимизаций},
	label={lst:clnoopt},
	style={go},
	linerange={7-19},
	]{../src/algs/algs.go}
\end{lstinputlisting}

\begin{lstinputlisting}[
	caption={Классический алгоритм умножения матриц с оптимизациями},
	label={lst:clopt},
	style={go},
	linerange={21-33},
	]{../src/algs/algs.go}
\end{lstinputlisting}
\newpage

В листингах \ref{lst:grnoopt}--\ref{lst:gropt} приведены реализации алгоритмов Винограда умножения матриц без оптимизаций и оптимизированного алгоритма Винограда умножения матриц.
\begin{lstinputlisting}[
	caption={Алгоритм Винограда умножения матриц без оптимизаций},
	label={lst:grnoopt},
	style={go},
	linerange={35-73},
	]{../src/algs/algs.go}
\end{lstinputlisting}

\pagebreak

\begin{lstinputlisting}[
	caption={Оптимизированный алгоритм Винограда умножения матриц},
	label={lst:gropt},
	style={go},
	linerange={196-236},
	]{../src/algs/algs.go}
\end{lstinputlisting}
\pagebreak

В листингах \ref{lst:shtnoopt}--\ref{lst:shtopt3} приведены реализации алгоритмов Штрассена умножения матриц без оптимизаций и оптимизированного алгоритма Штрассена умножения матриц.

\begin{lstinputlisting}[
	caption={Алгоритм Штрассена умножения матриц без оптимизаций},
	label={lst:shtnoopt},
	style={go},
	linerange={361-402},
	]{../src/algs/algs.go}
\end{lstinputlisting}

\begin{lstinputlisting}[
	caption={Продолжение листинга \ref{lst:shtnoopt}},
	label={lst:shtnoopt2},
	style={go},
	linerange={403-432},
	]{../src/algs/algs.go}
\end{lstinputlisting}

\begin{lstinputlisting}[
	caption={Оптимизированный алгоритм Штрассена умножения матриц},
	label={lst:shtopt},
	style={go},
	linerange={434-445},
	]{../src/algs/algs.go}
\end{lstinputlisting}

\begin{lstinputlisting}[
	caption={Продолжение листинга \ref{lst:shtopt}},
	label={lst:shtopt2},
	style={go},
	linerange={446-492},
	]{../src/algs/algs.go}
\end{lstinputlisting}

\begin{lstinputlisting}[
	caption={Продолжение листинга \ref{lst:shtopt2}},
	label={lst:shtopt3},
	style={go},
	linerange={493-499},
	]{../src/algs/algs.go}
\end{lstinputlisting}

Листинги со служебным кодом находятся в приложении.

\section{Функциональное тестирование}

В таблице \ref{tabular:functional_test} приведены функциональные тесты для алгоритмов умножения матриц: классического алгоритма умножения матриц, алгоритма Винограда умножения матриц, алгоритма Штрассена умножения матриц (неоптимизированная и оптимизированная версии). Все тесты пройдены успешно.

\begin{table}[h]
	\caption{\label{tabular:functional_test} Функциональные тесты}
	\begin{tabularx}{\textwidth}{|X|X|X|}
		\hline   
		\centering \textbf{Матрица 1} & \centering \textbf{Матрица 2} & \centering \textbf{Ожидаемый результат} \tabularnewline
		\hline
		\centering
		$\begin{pmatrix}
			1 & 2\\
			3 & 4\\
		\end{pmatrix}$ &
		\centering
		$\begin{pmatrix}
			5 & 6\\
			7 & 8\\
		\end{pmatrix}$ &
		\centering
		$\begin{pmatrix}
			19 & 22\\
			43 & 50\\
		\end{pmatrix}$\tabularnewline
		\hline
		\centering
		$\begin{pmatrix}
			10
		\end{pmatrix}$ &
		\centering
		$\begin{pmatrix}
			5
		\end{pmatrix}$ &
		\centering
		$\begin{pmatrix}
			50
		\end{pmatrix}$ \tabularnewline
		\hline
		\centering
		$\begin{pmatrix}
			10\\
			20\\
			30
		\end{pmatrix}$ &
		\centering
		$\begin{pmatrix}
			40 & 50 & 60\\
		\end{pmatrix}$ &
		\centering
		$\begin{pmatrix}
			400 &  500  & 600\\ 
			800 & 1000 & 1200\\ 
			1200 & 1500 & 180
		\end{pmatrix}$ \tabularnewline
		\hline
		\centering
		$\begin{pmatrix}
			1   &  2  &   3\\ 
			4   &  5  &   6
		\end{pmatrix}$ &
		\centering
		$\begin{pmatrix}
			1  &   2  &   3   &  4\\ 
			5  &   6  &   7   &  8\\ 
			9  &  10  &  11   & 12
		\end{pmatrix}$ &
		\centering
		$\begin{pmatrix}
			38 &   44  &  50  &  56\\ 
			83  &  98  & 113 &  128
		\end{pmatrix}$ \tabularnewline
		\hline
		%\noalign
	\end{tabularx}
\end{table}


\pagebreak

\section{Пример работы}

Демонстрация работы программы приведена на рисунке \ref{img:ex}.

\boximg{165mm}{ex}{Демонстрация работы программы}

\section*{Вывод}

В результате, были разработаны и протестированы следующие алгоритмы: классический алгоритм умножения матриц без оптимизаций, оптимизированный алгоритм умножения матриц, классический алгоритм Винограда, оптимизированный алгоритм Винограда, классический алгоритм Штрассена, оптимизированный алгоритма Штрассена
