\chapter*{ЗАКЛЮЧЕНИЕ}
\addcontentsline{toc}{chapter}{ЗАКЛЮЧЕНИЕ}

В ходе выполнения работы были выполнены все поставленные задачи:
\begin{itemize}
	\item[---] были исследованы алгоритмы умножения матриц;
	\item[---] была проведена оптимизация алгоритмов умножения матриц;
	\item[---] были применены методы динамического программирования для реализации алгоритмов умножения матриц;
	\item[---] были проведены оценка и сравнение трудоёмкости алгоритмов;
	\item[---] был проведён сравнительный анализ по времени и памяти на основе экспериментальных данных;
	\item[---] подготовлен отчет по лабораторной работе.
\end{itemize}

Исследование позволило выявить различия в производительности различных алгоритмов умножения матриц. В частности, алгоритм Винограда оказался самым эффективным среди реализованных (в 1.125 раз быстрее стандартного алгоритма умножения матриц, в 243 раза быстрее алгоритма Штрассена умножения матриц). 

Также была проведена оценка трудоёмкости данных алгоритмов на заданной модели вычислений. В результате наиболее трудоёмким оказался алгоритм Штрассена умножения матриц в неоптимизированной версии. Стандартный алгоритм и алгоритм Винограда умножения матриц имеют соизмеримые трудоёмкости. Оптимизации оказали положительное влияние на трудоёмкость реализаций алгритмов.

По потребляемой памяти также наиболее выигрышными оказались стандартный алгоримт умножения матриц и алгоритм Винограда умножения матриц. Алгоритм Штрассена потребляет значительно больше памяти, чем остальные алгоритмы за счёт своей рекурсивной реализации и использования дополнительных блочных подматриц.

В целом, исследование позволило получить практические и теоретические результаты, подтверждающие различия в производительности и использовании памяти различных алгоритмов умножения матриц. Эти результаты могут служить основой для выбора наиболее эффективного алгоритма в зависимости от конкретных требований и ограничений проекта.
\pagebreak