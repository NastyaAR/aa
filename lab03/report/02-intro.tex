\chapter*{ВВЕДЕНИЕ}
\addcontentsline{toc}{chapter}{ВВЕДЕНИЕ}

\textbf{Алгоритм сортировки}  --- это алгоритм упорядочивания элементов в коллекции. Поле, служащее критерием порядка, называется \textbf{ключом сортировки}.

Обычно алгоритмы сортировки классифицируют по времени работы, количеству требуемой дополнительной памяти, устойчивости, количеству обменов и детерминированности.

Алгоритмы сортировки являются фундаментальными инструментами компьютерной науки, которые позволяют упорядочить набор данных в определенной последовательности. Они играют важную роль в решении различных задач, от поиска элементов до оптимизации работы с данными.

Цель лабораторной работы ---  изучение, реализация и сравнение алгоритмов
сортировки. В данной лабораторной работе рассматриваются блосная сортировка, порязрядная сортировка и сортировка слиянием.

Задачи лабораторной работы:

\begin{itemize}
	\item[---] исследовать алгоритмы сортировки;
	\item[---] применить методы динамического программирования для реализации алгоритмов сортировки;
	\item[---] оценить и сравнить трудоёмкость алгоритмов сортировки;
	\item[---] провести сравнительный анализ по времени и памяти на основе экспериментальных данных;
	\item[---] подготовить отчет по лабораторной работе.
\end{itemize}

\pagebreak
