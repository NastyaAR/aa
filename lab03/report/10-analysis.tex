\chapter{Аналитическая часть}

\section{Блочная сортировка}

Блочная сортировка (корзинная сортировка) --- алгоритм сортировки, в котором сортируемые элементы распределяются между конечным числом отдельных блоков (карманов, корзин) так, чтобы все элементы в каждом следующем по порядку блоке были всегда больше
(или меньше), чем в предыдущем \cite{Theory}. 
Каждый блок затем сортируется отдельно либо рекурсивно тем же методом либо другим. Затем элементы помещаются обратно в массив.

Данный тип сортировки требует знания об устройстве входных данных. 
В случае их равномерного распределения алгоритм блочной сортировки может обладать линейным временем исполнения.

\section{Поразрядная сортировка}

Суть алгоритма поразрядной сортировки заключается в том, что массив несколько раз перебирается по разрядам и элементы перегруппировываются в зависимости от того, какая цифра находится в определённом разряде \cite{RadixTh}. 
При этом разряды могут обрабатываться в противоположных направлениях — от младших к старшим или наоборот.

После каждой сортировки по текущему разряду массив становится отсортирован по этому разряду. 
Для определения количества итераций необходимо предвычислять максимальное количество разрядов среди элементов массива.

\section{Сортировка слиянием}

Метод сортировки слияниями был предложен в 1945 году одним из величайших математиков ХХ века Джоном фон Нейманом.
Алгоритм основан на многократном слиянии уже упорядоченных и рядом расположенных групп элементов массива \cite{Merge}. 

Принцип «разделяй и властвуй» нашёл отражение в данной сортировке: массив разбивается на подмассивы меньшего размера, которые рекурсивно разбиваются на подмассивы. 
Процесс происходит до тех пор, пока в массиве не останется одного элемента --- такой случай тривиален, так как массив из одного элемента уже отсортирован. Затем происходит возврат из рекурсивных вызовов с объединением отсорированных частей массива.

\section*{Вывод}

Были рассмотрены алгоритмы алгоритмы сортировки: блочная сортировка, поразрядная сортировка, сортировка слиянием. 
Данные алгоритмы используют различные подходы к сортировке элементов массива. 
Блочная и поразрядная сортировки работают только с определённым типом входных данных, сортировка слиянием может работать с большим диапазоном типов входных данных.

\pagebreak

