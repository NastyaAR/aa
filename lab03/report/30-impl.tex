\chapter{Технологическая часть}

В данном разделе представлены требования к программному обеспечению, а также рассматриваются средства реализации и приводятся листинги кода.

\section{Требования к программному обеспечению}

К программе предъявляется ряд требований: на вход подаётся массив, содержащий целые числа, на выходе --- отсортированный массив, содержащий целые числа.

\section{Средства реализации}

В качестве языка программирования для реализации лабораторной работы был выбран многопоточный язык Gо \cite{golang}. Выбор был сделан в пользу данного языка программирования, вследствие наличия пакетов для тестирования программного обеспечения.

\section{Реализация}

В листингах \ref{lst:block1}--\ref{lst:block2} приведена реализация алгоритма блочной сортировки.
\newpage

\begin{lstinputlisting}[
	caption={Алгоритм блочной сортировки},
	label={lst:block1},
	style={go},
	linerange={24-71},
	]{../src/algs/algs.go}
\end{lstinputlisting}

\begin{lstinputlisting}[
	caption={Продолжение листинга \ref{lst:block1}},
	label={lst:block2},
	style={go},
	linerange={72-83},
	]{../src/algs/algs.go}
\end{lstinputlisting}

В листингах \ref{lst:radix1}--\ref{lst:radix2} приведена реализация алгоритма поразрядной сортировки.

\begin{lstinputlisting}[
	caption={Алгоритм поразрядной сортировки},
	label={lst:radix1},
	style={go},
	linerange={85-113},
	]{../src/algs/algs.go}
\end{lstinputlisting}

\begin{lstinputlisting}[
	caption={Продолжение листинга \ref{lst:radix1}},
	label={lst:radix2},
	style={go},
	linerange={114-143},
	]{../src/algs/algs.go}
\end{lstinputlisting}

В листингах \ref{lst:merge1}--\ref{lst:merge2} приведена реализация алгоритма сортировки слиянием.

\begin{lstinputlisting}[
	caption={Алгоритм сортировки слиянием},
	label={lst:merge1},
	style={go},
	linerange={145-155},
	]{../src/algs/algs.go}
\end{lstinputlisting}

\begin{lstinputlisting}[
	caption={Продолжение листинга \ref{lst:merge1}},
	label={lst:merge2},
	style={go},
	linerange={156-173},
	]{../src/algs/algs.go}
\end{lstinputlisting}

Листинги со служебным кодом находятся в приложении.

\section{Функциональное тестирование}

В таблице \ref{tabular:functional_test} приведены функциональные тесты для алгоритмов сортировки массивов целых чисел: алгоритма блочной сортировки, алгоритма поразрядной сортировки, алгоритма сортировки слиянием. Все тесты пройдены успешно.

\begin{table}[h]
	\caption{\label{tabular:functional_test} Функциональные тесты}
	\begin{tabularx}{\textwidth}{|X|X|}
		\hline   
		\centering \textbf{Массив} & \centering \textbf{Ожидаемый результат} \tabularnewline
		\hline
		\centering
		$\begin{matrix}
			1 & 2 & 3 & 4 & 5\\
		\end{matrix}$ &
		\centering
		$\begin{matrix}
			1 & 2 & 3 & 4 & 5\\
		\end{matrix}$\tabularnewline
		\hline
		\centering
		$\begin{matrix}
			5 & 4 & 3 & 2 & 1\\
		\end{matrix}$ &
		\centering
		$\begin{matrix}
			1 & 2 & 3 & 4 & 5\\
		\end{matrix}$\tabularnewline
		\hline
		\centering
		$\begin{matrix}
			1 & 1 & 1 & 1 & 1\\
		\end{matrix}$ &
		\centering
		$\begin{matrix}
			1 & 1 & 1 & 1 & 1\\
		\end{matrix}$\tabularnewline
		\hline
		\centering
		$\begin{matrix}
			1 & 23 & -3 & 94 & -517\\
		\end{matrix}$ &
		\centering
		$\begin{matrix}
			-517 & -3 & 1 & 23 & 94\\
		\end{matrix}$\tabularnewline
		\hline
		\centering
		$\begin{matrix}
			1\\
		\end{matrix}$ &
		\centering
		$\begin{matrix}
			1\\
		\end{matrix}$\tabularnewline
		\hline
		%\noalign
	\end{tabularx}
\end{table}


\pagebreak

\section{Пример работы}

Демонстрация работы программы приведена на рисунке \ref{img:ex}.

\boximg{130mm}{ex}{Демонстрация работы программы}

\section*{Вывод}

В результате, были разработаны и протестированы следующие алгоритмы: алгоритм блочной сортировки, алгоритм поразрядной сортировки, алгоритм сортировки слиянием.