\chapter*{ЗАКЛЮЧЕНИЕ}
\addcontentsline{toc}{chapter}{ЗАКЛЮЧЕНИЕ}

В ходе выполнения работы были выполнены все поставленные задачи:
\begin{itemize}
	\item[---] были исследованы алгоритмы сортировки;
	\item[---] были применины методы динамического программирования для реализации алгоритмов сортировки;
	\item[---] была произведена оценка и сравнение трудоёмкости алгоритмов сортировки;
	\item[---] был проведён сравнительный анализ по времени и памяти на основе экспериментальных данных;
	\item[---] был подготовлен отчет по лабораторной работе.
\end{itemize}

Исследование позволило выявить различия в производительности различных алгоритмов сортировки. В частности, реализация алгоритма поразрядной сортировки оказалась самой эффективной по времени исполнения наряду с реализацией алгоритма сортировки слиянием, который даёт примерно такие же результаты. Данные реализации алгоритмов работают примерно в 8 раз быстрее реализации алгоритма блочной сортировки на массиве длиной 610.

Также была проведена оценка трудоёмкости данных алгоритмов на заданной модели вычислений. В результате наиболее трудоёмким в среднем оказался алгоритм блочной сортировки. Алгоритм поразрядной сортировки имеет наименьшую трудоёмкость.

По потребляемой памяти наиболее выигрышным оказалась реализация алгоритма сортировки слиянием. Реализация алгоритма блочной сортировки потребляет наибольшее количество памяти вследствие выделения памяти под большое количество корзин, предусматриваемое алгоритмом.

В целом, исследование позволило получить практические и теоретические результаты, подтверждающие различия в производительности и использовании памяти различных алгоритмов сортировки. Эти результаты могут служить основой для выбора наиболее эффективного алгоритма в зависимости от конкретных требований и ограничений проекта.
\pagebreak